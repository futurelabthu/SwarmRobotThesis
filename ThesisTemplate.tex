% !TEX TS-program = xelatex
% !TEX encoding = UTF-8

% This is a simple template for a XeLaTeX document using the "ctexart" class by 张庭梁,
% with the fontspec package to easily select fonts.

\documentclass{ctexart} % 使用ctex 适配过的article 文档类
%\usepackage{fontspec} % Font selection for XeLaTeX; see fontspec.pdf for documentation
%\defaultfontfeatures{Mapping=tex-text} % to support TeX conventions like ``---''
%\usepackage{xunicode} % Unicode support for LaTeX character names (accents, European chars, etc)
%\usepackage{xltxtra} % Extra customizations for XeLaTeX
% other LaTeX packages.....
\usepackage{geometry} % See geometry.pdf to learn the layout options. There are lots.
\geometry{a4paper} % or letterpaper (US) or a5paper or....
%\usepackage[parfill]{parskip} % Activate to begin paragraphs with an empty line rather than an indent

\usepackage{graphicx} % support the \includegraphics command and options
\graphicspath {{fig/}} % 设置图片目录
\usepackage{cite}  %某一处需要同时引用多篇文献,在正文中直接引用   \cite{1}, \cite{2}   会生成这种样式:[1][2]   如果是3篇及以上    \cite{1,2,3}   会生成这种样式: [1-3]

\usepackage[cache=false]{minted}  %You must invoke LaTeX with the -shell-escape flag.
%如果出现一堆 Undefined control sequence.  那应该是没加 [cache=false]
%Package minted Error: You must have `pygmentize' installed to use this packag
%Solution:应该是Python的包Pygments的问题了,用pip安装了这个包
%   pip install pygments

\title{论文}
\author{张庭梁}
%\date{} % Activate to display a given date or no date (if empty),
         % otherwise the current date is printed 

\begin{document}
\maketitle

摘要:balabala、

关键词:\LaTeX

%\tableofcontents % 这里是目录

\section{第一节}

\begin{enumerate}
\item 
\item 
\item
\item
\end{enumerate}

如图~\ref{1}所示
% 如果想不浮动使用[H!]命令
\begin{figure}[htbp]
\small
\centering
\includegraphics[width=8cm]{1.jpg}
\caption{嘿嘿嘿} 
\label{1}
\end{figure}

%也可以直接将代码\include{}替换
%如果本身就是被include的 就使用  \input{data/TGAMcode}
%因为include 不能嵌套
\begin{minted}{c++}
int main() {
    printf("hello, world");
    return 0;
}
\end{minted}


%\section{结论}
%\section{致谢}
%\section{附录}

%% 参考文献
% 注意:至少需要引用一篇参考文献,否则下面两行可能引起编译错误。
% 如果不需要参考文献,请将下面两行删除或注释掉。
% 数字式引用
%要在文中汉字之后 \cite{论文名}  引用才会出现在参考文献中
%可以到Google scholar上找bibtex格式引用
\bibliographystyle{plain}
\bibliography{ref/refs}

\end{document}
